\chapter{はじめに}

%% 背景
近年、PC、サーバ、組込みシステム等、用途を問わずマルチプロセッサの利用
が進んでいる。その背景として、シングルプロセッサの高クロック化による性能
向上の限界や、消費電力・発熱の増大があげられる。マルチプロセッサシステム
では処理の並列性を高めることにより性能向上を実現するため、消費電力の増
加を抑えられる。組込みシステムにおいては、機械制御とGUIなど要件
の異なるサブシステム毎にプロセッサを使用する例があるなど、従来から複数
のプロセッサを用いるマルチプロセッサシステムが存在していたが、部品点数
の増加によるコスト増を招くため避けらていた。しかし、近年は、
1つのプロセッサ上に複数の実行コアを搭載したマルチコアプロセッサの登場に
より低コストで利用することが可能になり、低消費電力要件の強い、組込みシ
ステムでの利用が増加している。

マルチプロセッサ環境では、処理の並列性からプログラムの挙動が非決定的に
なり、タイミングによってプログラムの挙動が異なる。そのため、ブレークポ
イントやステップ実行を用いたシングルプロセッサ環境で用いられているデバッ
グ手法を用いることができない。

そのため、マルチプロセッサ環境では、プログラム実行履歴であるトレースロ
グの解析によるデバッグ手法が主に用いられる。トレースログを解析すること
で、各プロセスが、いつ、どのプロセッサで、どのような動作したかというプ
ログラムのデバッグに必要な情報が全て記録される。しかし、膨大な量となる
トレースログから特定の情報を探し出すのが困難である。さらに、各プロセッ
サのログが時系列に分散して記録されるため、逐次的にトレースログを解析す
ることも困難である。そのため、開発者が直接トレースログを解析するのは効
率が悪い。

トレースログの解析を支援するために、多くのトレースログ可視化ツールが開
発されている。組込みシステム向けデバッガソフトウェアや統合開発環境の一
部、Unix系OSのトレースログプロファイラなどが存在する。しかし、これら既
存のツールが扱うトレースログは、OSやデバッグハードウェアごとに異なるた
め、可視化対象が限定されおり、汎用性に乏しい。さらに、可視化表示項目が
提供されているものに限られ、追加や変更が容易ではなく、拡張性に乏しい。

そこで後藤ら\cite{goto,ipsj}によって汎用性と拡張性を備えたトレースログ
可視化ツール\textbf{TraceLogVisualizer (TLV)}が開発された。TLV内部でト
レースログを抽象的に扱えるよう、トレースログを一般化した標準形式トレー
スログを定め、任意の形式のトレースログを標準形式トレースログに変換する
仕組みを変換ルールとして形式化した。標準形式トレースログから図形データ
を生成する仕組みを抽象化し、可視化ルールとして形式化した。TLVでは、変換
ルールと可視化ルールを外部ファイルとして与えることで、汎用性と拡張性を
実現している。

本OJLでは、TLVの開発を継続して行なった。後藤らによって開発されたTLVのリ
リースを行ない、要求の収集を行なった。収集した要求のうち``CPU利用率表
示などの複雑な可視化の実現''と``TLVの高速化''に対する要求が強かったため、
これらの実現を行なった。

変換ルール、可視化ルールの機能は限定的であるため、CPU利用率などの複数の
ログをもとに図形を決定する必要のある複雑な可視化を行なえなかった。そこ
で、標準形式トレースログへの変換、及び図形データの生成を外部プロセスで行な
えるようにする。外部プロセスにすることで、
任意の言語で変換ルールと可視化ルールを記述できるため、
複雑な可視化を実現できる。

長時間動作するプログラムのトレースログは膨大な量となるため、トレースロ
グ解析を高速化する必要がある。しかし、標準形式トレースログへの変換及び
図形データの生成に関するソースコードが複雑化しているため、高速化のため
の変更を加えることが難しい。そこで、この問題を改善するために、関連する
ソースコードの調査し、複雑化している原因を特定し、それを改善するための
リファクタリング方針の決定を行なう。

%% 構成
最後に、本報告書の構成を述べる。
2章では、TLVの設計について述べる。
3章では、TLVの開発プロセスや発表実績などについて述べる。
4章では、複雑な可視化を行なうための機能追加について述べる。
5章では、高速化のためのリファクタリングについて述べる。
最後に6章で本論文のまとめと今後の展望と課題について述べる。

