\chapter{はじめに}

\section{開発背景}
近年,PC,サーバ,組み込みシステム等,用途を問わずマルチプロセッシングシステムの利用が進んでいる.
その背景には,シングルプロセッサの高クロック化による性能向上効果の停滞や,それに伴う消費電力・発熱の増大がある.
マルチプロセッシングシステムでは処理の並列性を高めることにより性能向上を実現するため,消費電力の増加を抑えることが出来る.
組み込みシステムにおいては,機械制御とGUIなど要件の異なるサブシステム毎にプロセッサを使用する例があるなど,従来から複数のプロセッサを用いるマルチプロセッサシステムが存在していたが,部品点数の増加によるコスト増を招くため避ける方向にあった.
しかしながら,近年は,1つのプロセッサ上に複数の実行コアを搭載したマルチコアプロセッサの登場により低コストで利用することが可能になり,低消費電力要件の強い組み込みシステムでの利用が増加している.

マルチコアプロセッサ環境でソフトウェアを開発する際に問題になる点として,デバッグの困難さが挙げられる.
これは,マルチコアプロセッサ環境では,従来のブレークポイントやステップ実行を用いたデバッグの挙動が,シングルプロセッサ環境における場合と異なるからである.
たとえば,複数のコアで実行される可能性のあるコードにブレークポイントを置きデバッグを行う場合,ブレーク後にステップ実行を行い処理を追う最中に,他のコアのブレークにより中断される可能性があるため,ブレーク後にブレークポイントを一時的に削除する必要があり,頻繁にブレークポイントの設置,削除を行わなければならなくなる.
また,マルチコアプロセッサ環境では,特殊な状況下でのみ起こるバグが発生する可能性がある.その場合,原因を特定するのは通常困難である.
なぜならば,マルチコアプロセッサ環境では,特別な制御を行わない限り各コアが非同期的に並列動作するため,バグの発生を再現することが難しいからである.
また,再現が可能であったとしても,原因を特定するためには,バグが発生するまでの過程をすべてのコアの実行状況を監視しながら行う必要があり,シングルプロセッサ環境の場合に比べ非常に煩雑になる.

一方,マルチコアプロセッサ環境で有効なデバッグ手法としてトレースログの解析によるデバッグがある.
これは,プログラムの実行中に,デバッグの判断材料となる情報をログとして出力することによりプログラムの動作を確認する手法である.
ログの出力元としては,OSやICE,シミュレータなどがあり,情報の粒度もアプリケーションレベル,タスクレベル,カーネルレベル,ハードウェアレベルなど様々である.

しかしながら,トレースログを開発者が直接扱うのは困難である場合が多い.
これは,ログの情報の粒度が細かくなるほど単位時間あたりのログの量が増える傾向にあり,膨大なログから所望の情報を探し出すのが困難であることや,各コアのログが時系列に分散して記録されるため,逐次的にログを解析することが困難であるからである.
そのため,トレースログの解析を支援するツールが要求されており,ログを解析し統計情報として出力したり,可視化表示することで開発者の負担を下げる試みが行われている.

既存のトレースログを可視化表示するツールとしては,ICE付属のデバッガソフトウェアや,各OSのシステムモニタリングツール,波形出力用ツールの流用などがある.
しかしながら,これらは,有料であったり,ターゲットOS,ターゲットCPUが限定されていたり,可視化表示項目,形式が固定であるなど,汎用性,拡張性の面で制限が多い.

こういった背景を鑑み,我々は,ログの形式非依存化,可視化表示項目のプラグイン化というアプローチで先例の制限を解決したトレースログ可視化ツール TraceLogVisualiser を開発した.

\section{開発目的}

\section{論文の構成}
