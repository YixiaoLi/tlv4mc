\chapter{おわりに}

\section{まとめ}

本論文では,トレースログ可視化ツールであるTLVの開発について,開発背景と既存ツールの問題点,設計と実装,利用例,また開発プロセスについて述べた.

TLVを開発するにあたり,動機となった背景として,組み込みシステムにおいてもマルチコアプロセッサの利用が進んでおり,それに伴い従来のデバッグ方法が有効でなくなってきたことを述べた.
これは,マルチコアプロセッサが各コアで並列処理を行うため,プログラムの挙動が非決定的になり,バグの再現が保証されず,従来のブレークポイントによるステップ実行ではバグを確実に捕らえることが出来ないからである.

一方,マルチコアプロセッサ環境におけるデバッグで有効な方法として,実行中にデバッグを行うのではなく,実行後にトレースログを解析する手法があることを述べた.
そして,開発者が直接トレースログを扱うのは効率が悪く,トレースログの解析を支援するツールが要求されており,その1つとして可視化表示ツールがあることを述べた.

既存のトレースログ可視化ツールは,標準化されたトレースログを扱っていないため,利用できるトレースログの形式が限られており,汎用性に乏しい点を指摘した.
また,可視化表示項目が提供されているものに限られ,変更や追加を行う仕組みも提供されておらず,拡張性に乏しい点についても述べた.

こういった既存のトレースログ可視化ツールの欠点を解消するため,汎用性と拡張性のあるトレースログ可視化ツールとしてTLVを開発することを目標として定めたことを述べた.

汎用性を確保するための解決策として,トレースログを一般化した標準形式トレースログを定義し,TLVがこの形式のトレースログの可視化に対応するようにしたことを述べた.
そして,任意の形式のトレースログは,標準形式トレースログへ変換することでTLVで扱うことが出来るようになることを述べた.
また,トレースログを変換する仕組みを形式化した,変換ルールファイルを定義することで変換が行える仕組みを提供したことも述べた.

実際に,シングルコアプロセッサ用RTOSのトレースログを標準形式トレースログへ変換し,可視化出来たことを示し有効性を確認した.
また,少量の変更でマルチコア用RTOSのトレースログの可視化に対応できることを示し,拡張性があることを示した.
最後に,RTOSのトレースログの形式とは全く異なる,組み込みコンポーネントシステムのトレースログを標準形式トレースログへ変換し可視化表示できることを示し,汎用性があることを示した.

次に,拡張性を確保するため実施したこととして,可視化表現とトレースログを対応付ける仕組みを形式化した,可視化ルールファイルを定義したことを述べた.
そして,可視化ルールファイルを記述することで,任意の可視化表示項目の追加や変更を行えることを述べた.

実際に,マルチコアプロセッサ用RTOSのトレースログを可視化する際に,所属プロセッサIDという可視化表示項目の追加を行ったが,新たにマルチコアプロセッサ用RTOSの可視化ルールファイルを追加するだけで実現出来たことを示し,拡張性があることを確認した.

TLVの開発は,OJL形式で行い,実際のソフトウェア開発現場で用いられているプロセスを適用して行ったことを述べた.
開発プロセスとしてユースケース駆動アジャイル開発を採用したことを述べ,各工程について作業内容を述べた,

\section{今後の課題と展望}

現在のTLVの課題としては次の2点がある.

1つ目の課題は,\ref{tecsRun}小節で述べたとおり,可視化表示をリソース毎に行で行っているため,行をまたぐ描画指定が出来ないことである.
これを解決する手段としては,可視化ルールファイルの図形定義の際に,図形を表示する行を指定させる方法が考えられる.

2つ目の課題は,計算や制御を伴う可視化表示が対応出来ない点である.
現在の変換ルールと可視化ルールは,置換マクロと条件を用いた指定は出来るようになっているが,変数を用いて状態を保持したり,任意回数ループして出力させたり,計算結果を用いて出力内容を変更するなどといったことが出来ない.
このため,ある一定期間のイベントの統計情報を用いるような可視化表示を行うことが出来ない.
たとえば,CPU使用率や,タスクのCPU占有率などである.
これを解決するには,可視化ルールや変換ルールをスクリプト言語を用いて記述出来るように拡張する方法が考えられる.

今後の展望としては,上記2点の課題の克服と,フェーズ3の実施による,対応するトレースログの拡充と,すでに可視化表示することが出来ているTOPPERS/ASP,FMPカーネル,TECSの可視化表示項目の追加を行う予定である.
現在のTOPPERS/ASP,FMPカーネルの可視化表示項目としては,リソースとしてタスクのみを考慮しているので,セマフォやイベントフラグなど,他のカーネルオブジェクトの可視化表示項目を追加することを予定している.
また,TECSに追加する可視化表示項目として,現在は,セルの呼び出し関係において,どんなシグネチャの呼び口の,どのメソッドを呼び出したかを考慮していないので,それらを考慮した可視化表現を考案したい.
