\chapter{はじめに}

\section{開発背景}
近年,PC,サーバ,組み込みシステム等,用途を問わずマルチプロセッシングシステムの利用が進んでいる.その背景には,シングルプロセッサの高クロック化による性能向上効果の停滞や,それに伴う消費電力・発熱の増大がある.マルチプロセッシングシステムでは処理の並列性を高めることにより性能向上を実現するため,消費電力の増加を抑えることが出来る.組み込みシステムにおいては,機械制御とGUIなど要件の異なるサブシステム毎にプロセッサを使用する例があるなど,従来から複数のプロセッサを用いるマルチプロセッサシステムが存在していたが,部品点数の増加によるコスト増を招くため避ける方向にあった.しかしながら,近年は,1つのプロセッサ上に複数の実行コアを搭載したマルチコアプロセッサの登場により低コストで利用することが可能になり,低消費電力要件の強い組み込みシステムでの利用が増加している.

\section{開発目的}

\section{論文の構成}
