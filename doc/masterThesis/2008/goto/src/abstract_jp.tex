近年,PC,サーバ,組み込みシステム等,用途を問わずマルチプロセッシングシステムの利用が進んでいる.
その背景には,シングルプロセッサの高クロック化による性能向上効果の停滞や,それに伴う消費電力・発熱の増大がある.
マルチプロセッシングシステムでは処理の並列性を高めることにより性能向上を実現するため,消費電力の増加を抑えることが出来る.
マルチコアプロセッサ環境でソフトウェアを開発する際に問題になる点として,デバッグの困難さが挙げられる.
これは,マルチコアプロセッサ環境では,従来のブレークポイントやステップ実行を用いたデバッグの挙動が,シングルプロセッサ環境における場合と異なるからである.

一方,マルチコアプロセッサ環境で有効なデバッグ手法としてトレースログの解析によるデバッグがある.
これは,プログラムの実行中に,デバッグの判断材料となる情報をログとして出力することによりプログラムの動作を確認する手法である.
しかしながら,トレースログを開発者が直接扱うのは困難である場合が多い.
これは,膨大なログから所望の情報を探し出すのが困難であることや,各コアのログが時系列に分散して記録されるため,逐次的にログを解析することが困難であるからである.
そのため,トレースログの解析を支援するツールが要求されている.

既存のトレースログ可視化表示ツールとしては,ICE付属のデバッガソフトウェアや,組み込みシステム向け統合開発環境の一部,カーネルトレースツールのGUI,波形出力用ツールの流用などがある.
しかし,これら既存のトレースログ可視化ツールは,ターゲットOSやターゲットCPUが制限されていたり,可視化表示項目が固定されていたりなど,汎用性や拡張性に乏しい.
そこで我々は,これらの問題点を解決し,汎用性,拡張性を備えたトレースログ可視化ツールを開発することを目的とし,TraceLogVisualizer (TLV) を開発した.また,その過程でトレースログの標準形式を提案した.
TLVの特徴は,既存のトレースログを標準形式トレースログに変換することによるログ形式の非依存化と可視化表示項目のプラグイン化である.

開発したTLVは,シングルコアプロセッサ用RTOSやマルチコアプロセッサ用RTOS,自動車制御用RTOS,組込みコンポーネントシステムなどの複数のトレースログの可視化表示を行うことで汎用性の確認を行い,可視化項目を目的に合わせてカスタマイズしたりプラグインとして追加出来ることを示し,拡張性があることを確認した.
また,TLVを複数の組み込みソフトウェア開発者に評価してもらい,マルチコアプロセッサ環境でのデバッグに有効であることを確認した.

TLVの開発はOJL(On the Job Learning)形式で行い,ユースケース駆動アジャイル開発というソフトウェア開発プロセスを用いて実施された.
