近年,組込みシステムにおいても,マルチプロセッサの利用が進んでいる.
その背景には,シングルプロセッサの高クロック化による性能向上効果の限界や,消費電力の増大がある.
マルチプロセッサシステムでは,処理の並列性を高めることにより性能向上を実現するため,消費電力の増加を抑えることができる.
しかし,マルチプロセッサ環境で動作するソフトウェアのデバッグは困難であるという問題がある.
これは,処理の並列性からプログラムの挙動が非決定的になり,バグの再現が保証されないため,シングルプロセッサ環境で用いられているブレークポイントやステップ実行を用いた従来のデバッグ手法が有効でないからである.

一方,マルチプロセッサ環境で有効なデバッグ手法として,プログラム実行履歴であるトレースログを解析する手法が挙げられる.
この手法が有効である理由は,並列プログラムのデバッグにおいて必要な情報である,各プロセスが,いつ,どのプロセッサで,どのように動作していたかということを,トレースログを解析することで知ることができるからである.
しかしながら,開発者が直接トレースログを解析するのは効率が悪いという問題がある.
これは,膨大な量となるトレースログから所望の情報を探し出すのが困難であることや,各プロセッサのログが時系列に分散して記録されるため,逐次的にログを解析することが困難であることが理由である.

トレースログの解析を支援する方法として,ツールによるトレースログの可視化が挙げられ,これまでに多くのトレースログ可視化ツールが開発されている.
具体的には,組込みシステム向けデバッガソフトウェアや統合開発環境の一部,Unix系OSのトレースログプロファイラなどが存在する.
しかしながら,これら既存のツールが扱うトレースログは,形式が標準化されておらず,環境(OSやデバッグハードウェア)毎に異なるため,可視化対象が限定されおり,汎用性に乏しい.
さらに,可視化表示項目が提供されているものに限られ,追加や変更が容易ではないなど,拡張性に乏しいといった問題もある.

そこで我々は,これらの問題点を解決し,汎用性と拡張性を備えたトレースログ可視化ツールを開発することを目的とし,TraceLogVisualizer (TLV) を開発した.
まず,TLV内部でトレースログを抽象的に扱えるよう,トレースログを一般化した標準形式トレースログを定め,任意の形式のトレースログを標準形式トレースログに変換する仕組みを変換ルールとして形式化した.
次に,トレースログの可視化表現を指示する仕組みを抽象化し,可視化ルールとして形式化した.
TLVでは,変換ルールと可視化ルールを外部ファイルとして与えることで,汎用性と拡張性を実現した.

開発したTLVを用いて,シングルコアプロセッサ用RTOS(Real-time operating system)やマルチコアプロセッサ用RTOS,組込みコンポーネントシステムなど,形式が異なる様々なトレースログの可視化を試み汎用性の確認を行った.
また,可視化表示項目の変更,追加を試み拡張性の確認を行った.
その結果,変換ルールと可視化ルールの変更,追加でこれらが実現可能であることを示した.

TLV の開発は,OJL(On the Job Learning) 形式で行い,開発プロセスに,ユースケース駆動アジャイル開発を適用して実施した.
