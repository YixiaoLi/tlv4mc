In recent years, multiprocessors have been used even in embedded systems.
The reason for this is limits of effectiveness that improve a performance through overclocking in a single processor.
A multiprocessor system can reduce the increase in power consumption even though improving performance by parallel processing.
However, there is a problem that it is difficult to debug software executed in multiprocessors.
This means that a traditional debugging way using breakpoints and step excuting is ineffective, because a repeatability of bug is made low by a nondeterministic behavior of parallel processing.

On the other hand, an effective technique to debug software excuted in multiprocessors includes analyzing a trace log that is a program execution history.
The reason for that the technique is effective is that it can know necessary information for a parallel program debugging by analyzing the trace log.
The information, for example, are when, where or how long processes executed.
However, it is inefficient that developeres analyze the trace log directly, because searching a desired information in tremendous quantities of trace log and analyzing time-series trace log recorded by a number of processors sequentially are difficult.

Techniques to support developeres to analyze the trace log includes visualizing trace log by tools.
And, many visualizing tools for trace log have been developed before now.
In particular, there are debugging software or integrated development environment for embedded systems and trace log profilers for Unix-like operating systems.
However, these existing tools lack general versatility because they treat only own format trace log. 
And, they lack expandability because visualized information are limited to items provided by them.

To that end, we developed TraceLogVisualizer(TLV), a visualization tool forthe trace log, for the purpose of implementation of general versatility and expandability.
First, we defined the standard-format-trace-log with generalizing a trace log.
This is necessity for that the TLV treat the trace log abstractively.
And, we provided a mechanism that any format trace logs convert to the standard-format-trace-log by writing a convert-rule-file.
Next, we formalized a mechanism associating shapes and trace logs as visualize-rule-file through abstracting them.

We confirmed general versatility through attemptting to visualize trace logs of a wide variety of formats including the trace log of RTOS (Real-time operating system) for a singlecore processor and multicore processors, and embedded component systems.
Also, we confirmed expandability through attemptting to add and change visualized information.
In the result, they are achieved by writing convert-rule-files and visualize-rule-files.

Development of TLV was performed as OJL(On the Job Learning), and development process carried out by applying use case driven agile development.
